\documentclass{article}
\usepackage{amsmath}
\usepackage{amsfonts}
\usepackage{amssymb}
\newcommand*{\E}{\mathbb{E}}
\newcommand*{\I}{\mathbf{I}}

\begin{document}

\section{VAR Setup}

Let a structural VAR be of the following form:
\begin{align*}
  B_0 y_t &= B_1 y_{t-1} + \ldots + B_p y_{t-p} + w_t
\end{align*}

where
$y_t$ is a $K \times 1$ vector of endogenous variables,
$B_i$ are a $K \times K$ matrix of coefficients on the endogenous variables,
$w_t$ is a $K \times 1$ vector of exogenous structural shocks.

But we can only observe a reduced form VAR:
\begin{align*}
  y_t &= A_1 y_{t-1} + \dots + A_p y_{t-p} + u_t
\end{align*}

where
$A_i$ are a $K \times K$ matrix of coefficients on the endogenous variables,
$u_t$ is a $K \times 1$ vector of reduced form shocks.

We assume without any loss of generality that 
\begin{align*}
  \E[w_t w_t^\prime] = \I
\end{align*}

\subsection{VAR(1) form}
It is useful to define a VAR(1) form for any VAR(p) process,

\begin{align*}
  Y_t &= \mathbf{A} Y_{t-1} + U_t
\end{align*}

where
\begin{align*}
  Y_t &= \begin{bmatrix} y_t \\ y_{t-1} \\ \vdots \\ y_{t-p+1} \end{bmatrix}, \
  \mathbf{A} = \begin{bmatrix} A_1 & A_2 & \dots& A_p\\ \I & 0 & \dots & 0 \\ 0 & \I & \dots & 0 \\ \vdots & \vdots & \ddots & \vdots \\ 0 & 0 & \dots & \I \end{bmatrix}, \
  U_t = \begin{bmatrix} u_t \\ 0 \\ \vdots \\ 0 \end{bmatrix}
\end{align*}

And define the matrix $J$ as the following for recovering the original dimensions:
\begin{align*}
  \underbrace{J}_{K \times K\cdot p} \equiv \begin{bmatrix} \mathbf{I}_K & 0_{K \times K(p-1)} \end{bmatrix}
\end{align*}

Then we have that,
\begin{align*}
  J Y_t = y_t, \ \ \
  J U_t = u_t, \ \ \
  J J^\prime = \I_K
\end{align*}

\section{Identification Problem}

The identification problem is to identify a structural VAR using only reduced form VAR observations.
This amounts to identifying the inverse of the impact matrix $B_0$.
This matrix provides a linear mapping between structural and reduced form shocks.

\begin{align*}
  u_t &= B_0^{-1} w_t
\end{align*}

\begin{align*}
  \begin{bmatrix} u_t^1 \\ u_t^2 \\ \vdots \\ u_t^K \end{bmatrix}
  =
  \begin{bmatrix}
    b_0^{11} & b_0^{12} & \dots & b_0^{1K} \\
    b_0^{21} & b_0^{22} & \dots & b_0^{2K} \\
    \vdots & \vdots & \ddots & \vdots \\
    b_0^{K1} & b_0^{K2} & \dots & b_0^{KK}  \end{bmatrix}
  \begin{bmatrix} w_t^1 \\ w_t^2 \\ \vdots \\ w_t^K \end{bmatrix}
\end{align*}

This, along with the assumption about the unit variance of structural shocks,implies that

\begin{align*}
  \E[u_t u_t^\prime] = \Sigma_u = B_0^{-1} B_0^{-1\prime}
\end{align*}

\section{Cholesky Identification}

The Cholesky identification scheme is a recursive scheme that identifies the impact matrix $B_0$ by assuming that the shocks are ordered in the way given in $y_t$.

This is equivalent to the following set of 0 resetrictions on the impact matrix $B_0^{-1}$:

\begin{align*}
  \begin{bmatrix} u_t^1 \\ u_t^2 \\ \vdots \\ u_t^K \end{bmatrix}
  =
  \begin{bmatrix}
    b_0^{11} & 0 & \dots & 0 \\
    b_0^{21} & b_0^{22} & \dots & 0 \\
    \vdots & \vdots & \ddots & \vdots \\
    b_0^{K1} & b_0^{K2} & \dots & b_0^{KK}
  \end{bmatrix}
  \begin{bmatrix} w_t^1 \\ w_t^2 \\ \vdots \\ w_t^K \end{bmatrix}
\end{align*}

These $b_0^{ij}$ are the elements of the Cholesky of the covariance matrix of the structural shocks.

\begin{align*}
  \widehat{B}_0^{-1} = \text{chol} (\widehat{\Sigma}_u)
\end{align*}

\section{Impulse Response Functions}

\subsection{Reduced Form IRFs}

Reduce form IRFs are defined as the following object:
\begin{align*}
  \Phi_i \equiv J \mathbf{A}^i J^\prime
\end{align*}

This implies that the first reduced form IRF is the identity matrix,
\begin{align*}
  \Phi_0 &= \I_K
\end{align*}

\subsection{Structural IRFs}

The structural IRFs are defined as the following object:

\begin{align*}
  \frac{\partial y_{t+i}}{\partial w_t} = \Theta_i \hspace{1cm} i = 0, 1, 2, \dots, H
\end{align*}

and each element of the $K \times K$ matrix, $\Theta_i$, is the response of variable $j$ to shock $k$ at time $i$,

\begin{align*}
  \theta^{jk}_{i} = \frac{\partial y_{t+i}^j}{\partial w_t^k}
\end{align*}

Start from the moving average (MA) representation of the VAR process,
\begin{align*}
  y_t &= \sum_{i=0}^\infty \Phi_i u_{t-i}
\end{align*}

This can be rewritten in terms of the structurual IRFs and shocks,

\begin{align*}
  y_t &= \sum_{i=0}^\infty \Phi_i B_0^{-1} B_0 u_{t-i} 
  \\
  &= \sum_{i=0}^\infty \Theta_i w_{t-i}
\end{align*}

This shows we can convert from reduced form IRFs to structural IRFs by post multiplying by the inverse of the impact matrix,

\begin{align*}
  \Theta_0 &= \Phi_0 B_0^{-1} = B_0^{-1}
  \\
  \Theta_1 &= \Phi_1 B_0^{-1}
  \\
  \Theta_2 &= \Phi_2 B_0^{-1}
  \\
  \vdots
\end{align*}

Comparing just the immediate IRFs to the impact matrix, we see that

\begin{align*}
  \begin{bmatrix}
    \theta_0^{11} & 0 & \dots & 0 \\
    \theta_0^{21} & \theta_0^{22} & \dots & 0 \\
    \vdots & \vdots & \ddots & \vdots \\
    \theta_0^{K1} & \theta_0^{K2} & \dots & \theta_0^{KK}
  \end{bmatrix}
  =
  \begin{bmatrix}
    b_0^{11} & 0 & \dots & 0 \\
    b_0^{21} & b_0^{22} & \dots & 0 \\
    \vdots & \vdots & \ddots & \vdots \\
    b_0^{K1} & b_0^{K2} & \dots & b_0^{KK}
  \end{bmatrix}
\end{align*}

Which shows that for the first shock, $w_t^k$, we have $K$ responses, all in the first column of $\Theta_0$.

This is the identifying assumption of a Cholesky decomposition: the first shock has a contemporaneous effect on all variables, but no other contemporary variables affect it.



\end{document}