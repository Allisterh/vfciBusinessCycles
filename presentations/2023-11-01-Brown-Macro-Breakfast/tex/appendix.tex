\begin{frame}
  \frametitle{}
  \centering
  \Huge
  APPENDIX
\end{frame}

\begin{frame}{Shock Weights}

  \begin{figure}
    \includegraphics[height = 3in]{figs/results-fig4-vfci_u_0632_qb.pdf}
  \end{figure}

\end{frame}

%%%%%%%%%%%%%%%%%%%%%%%%%%%%%%%%%%%%%%%

\begin{frame}{Shock Weights: 18 to 36 q}

  \begin{figure}
    \includegraphics[height = 3in]{figs/results-fig54-vfci_u_1836_qb.pdf}
  \end{figure}

\end{frame}



\begin{frame}{Historical Contribution: 18 to 36 q}

  \begin{figure}
    \includegraphics[height = 3in]{figs/results-fig52-vfci_u_1836_hd.pdf}
  \end{figure}

\end{frame}


\begin{frame}{FEVDFD: 18 to 36 q}

  \begin{figure}
    \includegraphics[height = 3in]{figs/results-fig53-vfci_u_1836_fevdfd.pdf}
  \end{figure}

\end{frame}

%%%%%%%%%%%%%%%%%%%%%%%%%%%%%%%%%%%%%%%

\begin{frame}{Comparing VAR Targets}

  \begin{figure}
    \includegraphics[height = 3in]{figs/results-fig10-compare_var_targets_orth.pdf}
  \end{figure}

\end{frame}



\begin{frame}{Comparing VAR Targets}

  \begin{figure}
    \includegraphics[height = 3in]{figs/results-fig11-compare_var_target_fevdfd.pdf}
  \end{figure}

\end{frame}

%%-- Inflation and Interest Rate charts

\begin{frame}{Targeting Inflation and Interest Rate}

  \begin{figure}
    \includegraphics[height = 3in]{figs/mp-fig21-inf_int_irf.pdf}
  \end{figure}

\end{frame}


\begin{frame}{Targeting Inflation and Interest Rate}

  \begin{figure}
    \includegraphics[height = 3in]{figs/mp-fig22-inf_int_hd.pdf}
  \end{figure}

\end{frame}



\begin{frame}{Targeting Inflation and Interest Rate}

  \begin{figure}
    \includegraphics[height = 3in]{figs/mp-fig23-inf_int_fevdfd.pdf}
  \end{figure}

\end{frame}

\begin{frame}{Targeting Inflation and Interest Rate}

  \begin{figure}
    \includegraphics[height = 3in]{figs/mp-fig24-inf_int_qb.pdf}
  \end{figure}

\end{frame}

%%%%%%%%%%%%%%%%%%%%%%%%%%%%%%%%%%%%%%%%%%%%%%%%%%


\begin{frame}{VFCI Estimation}
  \label{vfci_estimation}
  
  \vspace{0.3cm}
  Assuming (1) no arbitrage and (2) a representative agent,
  
  a log-linear approximation of the representative agent's FOC relates
  \begin{itemize}
    \item asset prices, $R_t$
    \item future consumption growth, $\Delta c_{t+1} = \ln c_{t+1} - \ln c_t$
  \end{itemize}

  This relationship can be estimated empirically:

  \vspace{-1cm}
  \begin{align}
    \Delta c_{t+1} &= \beta R_t + \varepsilon_t
    \\
    \ln \text{Vol}[\varepsilon_t] &= \lambda R_t + \upsilon_t
  \end{align}

  VFCI is predicted value from eq. (2). Interpreted as the ``price of risk''.

  \vspace{-1cm}
  \setcounter{equation}{0}
  \begin{align*}
    \text{VFCI}_t &\equiv \ln \widehat{\text{Vol}[\epsilon_t]}
  \end{align*}

  \returnbutton{1}{0.3}{vfci}{VFCI}
\end{frame}

\begin{frame}{VFCI Estimation}

  \vspace{0.3cm}
  Assuming (1) no arbitrage and (2) a representative agent,
  
  a log-linear approximation of the representative agent's FOC relates
  \begin{itemize}
    \item asset prices, $R_t$
    \item future consumption growth, $\Delta c_{t+\textcolor{red}{2}} = \ln c_{t+\textcolor{red}{2}} - \ln c_{t\textcolor{red}{+1}}$
  \end{itemize}

  This relationship can be estimated empirically:

  \vspace{-1cm}
  \begin{align}
    \Delta c_{t+1} &= \beta R_t + \varepsilon_t
    \\
    \ln \text{Vol}[\varepsilon_t] &= \lambda R_t + \upsilon_t
  \end{align}

  VFCI is predicted value from eq. (2). Interpreted as the ``price of risk''.

  \vspace{-1cm}
  \setcounter{equation}{0}
  \begin{align*}
    \text{VFCI}_t &\equiv \ln \widehat{\text{Vol}[\epsilon_t]}
  \end{align*}

  \textcolor{red}{Consumption forwarded one more period to align with theory.}

  
\end{frame}


\begin{frame}{VFCI Estimation}

  \vspace{0.3cm}
  Assuming (1) no arbitrage and (2) a representative agent,
  
  a log-linear approximation of the representative agent's FOC relates
  \begin{itemize}
    \item asset prices, $R_t$
    \item future consumption growth, $\Delta c_{t+1+\textcolor{blue}{h}} = \ln c_{t+ 1 + \textcolor{blue}{h}} - \ln c_{t+1}$
  \end{itemize}

  This relationship can be estimated empirically:

  \vspace{-1cm}
  \begin{align}x
    \Delta c_{t+1} &= \beta R_t + \varepsilon_t
    \\
    \ln \text{Vol}[\varepsilon_t] &= \lambda R_t + \upsilon_t
  \end{align}

  VFCI is predicted value from eq. (2). Interpreted as the ``price of risk''.

  \vspace{-1cm}
  \setcounter{equation}{0}
  \begin{align*}
    \text{VFCI}_{t,\textcolor{blue}{h}} &\equiv \ln \widehat{\text{Vol}[\epsilon_t]}
  \end{align*}

  \textcolor{blue}{Can also consider longer forward growth horizons, $h \in [1,\infty)$.}

\end{frame}


%%%%%%%%%%%%%%%%%%%%%%%%%%%%%%%%%%%%%%%%%%

\begin{frame}{VAR Setup}

  \begin{enumerate}
    \item Run the empirical VAR
  
    \vspace{-1cm}
    \begin{align*}
      A(L) X_t &= u_t
    \end{align*}
    \vspace{-1cm}
  
    with \ \ \ $A(L) \equiv \sum_{\tau = 0}^{p} A_\tau L^\tau$, \ \ \ $A_0 = I$, \ \ \ and \ \ \ $\mathbf{E}[u_t u_t'] = \Sigma$
  
    \item Orthogonalize the residuals, $S = $ Choleskey decomposition of $\Sigma$
    
    \vspace{-1cm}
    \begin{align*}
      u_t = S \epsilon_t
    \end{align*}
    \vspace{-1cm}
  
    with \ \ \ $\mathbf{E}[\epsilon_t \epsilon_t'] = I$
  
    \item Denote all possible rotations, $Q$, of structural shocks, $\epsilon_t$
    
    \vspace{-1cm}
    \begin{align*}
      S = \tilde{S}Q
    \end{align*}
    \vspace{-1cm}
  
    where $Q$ is any orthonormal ($QQ' = I$) rotation matrix.
  
    \item This is the identification problem. Which $Q$ to choose?
  \end{enumerate}
  
  \end{frame}
  
  
  \begin{frame}{Max Variance Identification}
  
    \begin{enumerate}
      \item Write out the VMA($\infty$) representation of a VAR($p$)
    
      \vspace{-1cm}
      \begin{align*}
        X_t &= B(L) u_t
      \end{align*}
      \vspace{-1cm}
    
      with \ \ \ $B(L) \equiv \sum_{\tau = 0}^{p} B_\tau L^\tau$ \ \ \ and \ \ \ $B(L) = A(L)^{-1}$
    
      \item Substitute in rotations of structural shocks, $u_t = \tilde{S}Q\epsilon_t$
      
      \vspace{-1cm}
      \begin{align*}
        X_t = C(L) Q \epsilon_t
      \end{align*}
      \vspace{-1cm}
    
      with \ \ \ $C(L) = B(L)\tilde{S}$ \ \ \ and \ \ \ $\Gamma(L) = C(L) Q$  \ \ \ stores the IRF.
    
      
    \end{enumerate}
    
    \end{frame}
  
  
    \begin{frame}{Max Variance Identification}
  
      \begin{enumerate}
        \item The forecast error variance (FEV) for time horizon of $T$
      
      \vspace{-1cm}
      \begin{align*}
        FEV_{T} &= \sum_{t=0}^{T} \Gamma(t)'\Gamma(t)
        \\
        &= \sum_{t=0}^{T} Q'C(t)'C(t)Q
      \end{align*}
      \vspace{-1cm}
      
        \item The forecast error variance for a frequency range, $\{\underline{\omega}, \bar{\omega}\}$
        
        \vspace{-1cm}
        \begin{align*}
          FEV_{\underline{\omega}, \bar{\omega}} &= \int_{\underline{\omega}}^{\bar{\omega}} Q'C(e^{-i\omega})'C(e^{-i\omega})Q \ d\omega
          \\
          &= Q' \left( \int_{\underline{\omega}}^{\bar{\omega}} C(e^{-i\omega})'C(e^{-i\omega}) \ d\omega \right) Q
        \end{align*}
        \vspace{-1cm}
      
        \dots
      \end{enumerate}
      
      \end{frame}